\documentclass[paper=a4, fontsize=11pt]{article} % A4 paper and 11pt font size

\usepackage[T1]{fontenc} % Use 8-bit encoding that has 256 glyphs
\usepackage[english]{babel} % English language/hyphenation
\usepackage{amsmath,amsfonts,amsthm} % Math packages
\usepackage{a4wide}
\usepackage{float}
\usepackage{longtable}
\usepackage{hyperref}
\usepackage{listings}
\usepackage{makecell}
\usepackage[table]{xcolor}
\lstset{deletestring=[b]{"}}

\usepackage{fancyhdr} % Custom headers and footers
\pagestyle{fancyplain} % Makes all pages in the document conform to the custom headers and footers
\fancyhead[L]{SF3565, Project 2, October 2016}
\fancyhead[R]{H{\"a}ggmark, Karlsson} % Empty left footer
\fancyfoot[C]{Program construction in C++ for Scientific Computing} % Empty center footer
\fancyfoot[R]{\thepage} % Page numbering for right footer

\title{Program construction in C++ for Scientific Computing \\ Teacher: Michael Hanke}

\author{Ilian H{\"a}ggmark \\ mail \href{mailto:ilianh@kth.se}{ilianh@kth.se}
  \and Andreas Karlsson \\ mail \href{mailto:andreas.a.karlsson@ki.se}{andreas.a.karlsson@ki.se} }
\date{\normalsize\today} % Today's date or a custom date

\begin{document}

\maketitle % Print the title

\section{Project 2}
\subsection{Task 1 - implement the evaluation of the exponential for real numbers.}
\label{subsec:task1}

The exponential for real number is based on a simple Taylor expansion
and the implementation uses Horner's scheme to improve the
performance.  To estimate the error a for-loop calculate each term in
the Taylor expansion. If term $n$ is smaller than the error the sum of
consecutive terms is smaller than the term $n$ and thus the
error. Horner's scheme then uses $n$-terms for the approximation.

\small{
  \begin{table}[H]
    \begin{tabular}{ l | l l l l l l}
      \diaghead{Tolerance}{tol}{x} & -1 & 1 & 3 & 5 & 10 & 50 \\
      \hline
      0.01 & 0.00121277  & 0.00161516  & 0.00143384  & 0.00294883 & 0.00175855 & \cellcolor{red!20}3.14573e+06\\
      0.001 & 2.22983e-05 & 2.78602e-05 & 6.83293e-05 & 0.00020803 & 0.000162384 & \cellcolor{red!20}3.14573e+06\\
      1e-08 & 1.49839e-10 & 1.72876e-10 & 1.67e-09 & 8.31676e-10 & 9.24047e-10 & \cellcolor{red!20}3.14573e+06\\
      1e-10 & 7.19536e-13 & 8.15348e-13 & 4.16023e-12 & 4.17799e-12 & 1.45519e-11 & \cellcolor{red!20}3.14573e+06\\
    \end{tabular}
    \caption{ Absolute difference between the \lstinline$cmath$
      implementation and our implementation of an exponential function
      using Taylor expansion and Horner's scheme for a number of $x$
      values and tolerances. \ref{tab:p2t1-errors}.}
    \label{tab:p2t1-errors}
  \end{table}
} The code is tested with a set of tolerances and exponents $x$. As
can be seen in Table~\ref{tab:p2t1-errors} the absolute difference
with the \texttt{cmath} \texttt{exp} function is within the set
tolerance for low $x$ values. However for $x$ larger than 10 the error
exceeds the set tolerance. This is because the type \texttt{double}
cannot store data with high enough precision. A \texttt{double} can
store about 15 digits. If the desiered error should be $10^{-10}$ then
the largest value that can be stored with this precision is
$\sim 10^{5}$.


\subsection{Task 2 - construct a matrix class.}

A basic matrix class has been constructed that contains the basic
constructors and operator overloading that are required to calculate
the matrix exponential with Taylor expansion and Horner's scheme,
i.e., the same way as in Section~\ref{subsec:task1}.

The matrix object contains an \texttt{int} with the size of the matrix
and a C++ vector object that contains the matrix values. The approach
to determine the required number of terms in the Taylor expansion to
get an error below a set tolerance is equivalent to that in
Section~\ref{subsec:task1}. Just as for Section~\ref{subsec:task1}
this works well for matrices with small elements but not for large
elements since the result is so large that the type \texttt{double}
cannot contain the values with enough precision.

In order to test the \texttt{matrixExp} function we compared it with
Matlab's \texttt{r8mat\_{expm1}} function. We started by creating a
\texttt{Matrix} class objects called $M_{\text{input}}$.

\[
M_{\text{input}}=
  \begin{bmatrix}
    1 & 3 & 10 & 45 \\
    12 & 3 & 5 & 0 \\
    12 & 1 & 3 & 7 \\
    19 & 4 & 9 & 6 \\
  \end{bmatrix}
\]

We then used our \texttt{matrixExp} function to calculate the
exponential of the matrix $E_{\text{our}}$.

\[
E_{\text{our}} = \mathtt{matrixExp(M,0.001)} =
  \begin{bmatrix}
 8.47678e+16 & 2.18816e+16 & 5.62372e+16 & 1.24171e+17 \\
 3.30298e+16 & 8.52616e+15 & 2.19129e+16 & 4.83831e+16 \\
 4.02549e+16 & 1.03912e+16 & 2.67062e+16 & 5.89667e+16 \\
 6.21118e+16 & 1.60332e+16 & 4.12066e+16 & 9.09833e+16 \\
  \end{bmatrix}
\]

We also calculated the exponential of the matrix using Matlabs
\texttt{r8mat\_{expm1}} function, which produces the matrix displayed
as $E_{\text{matlab}}$.

\[
E_{\text{matlab}} = \mathtt{r8mat\_{expm1}(4, mArray)} =
  \begin{bmatrix}
 8.47678e+16 & 2.18816e+16 & 5.62372e+16 & 1.24171e+17 \\
 3.30298e+16 & 8.52616e+15 & 2.19129e+16 & 4.83831e+16 \\
 4.02549e+16 & 1.03912e+16 & 2.67062e+16 & 5.89667e+16 \\
 6.21118e+16 & 1.60332e+16 & 4.12066e+16 & 9.09833e+16 \\
  \end{bmatrix}
\]

We than compared the two functions by calculating the difference which
is displayed as $E_{\text{difference}}$.


\[
  E_{\text{difference}} = E_{\text{our}} - E_{\text{matlab}} =
\begin{bmatrix}
 672 & 144 & 448 & 784 \\
 208 & 45 & 140 & 248 \\
 320 & 70 & 216 & 400 \\
 424 & 92 & 288 & 512 \\
\end{bmatrix}
\]

\end{document}
