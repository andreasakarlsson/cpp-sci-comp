\documentclass[paper=a4, fontsize=11pt]{article} % A4 paper and 11pt font size

\usepackage[T1]{fontenc} % Use 8-bit encoding that has 256 glyphs
\usepackage[english]{babel} % English language/hyphenation
\usepackage{amsmath,amsfonts,amsthm} % Math packages
\usepackage{a4wide}
\usepackage{float}
\usepackage{longtable}
\usepackage{hyperref}
\usepackage{listings}
\lstset{deletestring=[b]{"}}

\usepackage{fancyhdr} % Custom headers and footers
\pagestyle{fancyplain} % Makes all pages in the document conform to the custom headers and footers
\fancyhead[L]{SF3565, Project 2, October 2016}
\fancyhead[R]{H{\"a}ggmark, Karlsson} % Empty left footer
\fancyfoot[C]{Program construction in C++ for Scientific Computing} % Empty center footer
\fancyfoot[R]{\thepage} % Page numbering for right footer

\title{Program construction in C++ for Scientific Computing \\ Teacher: Michael Hanke}

\author{Ilian H{\"a}ggmark \\ mail \href{mailto:ilianh@kth.se}{ilianh@kth.se}
  \and Andreas Karlsson \\ mail \href{mailto:andreas.a.karlsson@ki.se}{andreas.a.karlsson@ki.se} }
\date{\normalsize\today} % Today's date or a custom date

\begin{document}

\maketitle % Print the title

\section{Project 2}
\subsection{Task 1 - implement the evaluation of the exponential for real numbers.}

The exponential for real number is based on a simple Taylor expansion and the implementation uses Horner’s scheme to improve the performance.
To estimate the error a for-loop calculate each term in the Taylor expansion. If term $n$ is smaller than the error the sum of consecutive terms is smaller than the term $n$ and thus the error. Horner’s scheme then uses $n$-terms for the approximation.
The code is tested with a set of tolerances and exponents $x$. The function works well for low $x$, but when $x$ is over 10 the error exceeds the set tolerance. This is because double type cannot store data with high enough precision. A double can store about 15 digits. If the desiered error should be $10^{-10}$ then the largest value that can be stored with this precision is $\sim 10^{5}$.


\subsection{Task 2 - construct a matrix class.}




\end{document}
