\documentclass[paper=a4, fontsize=12pt]{article} % A4 paper and 11pt font size
\usepackage[T1]{fontenc} % Use 8-bit encoding that has 256 glyphs
\usepackage[english]{babel} % English language/hyphenation
\usepackage{amsmath,amsfonts,amsthm} % Math packages
\usepackage{a4wide}
\usepackage{float}
\usepackage{longtable}
\usepackage{hyperref}
\usepackage{listings}
\usepackage{makecell}
\usepackage{graphicx}
\usepackage[table]{xcolor}
\usepackage[numbered, framed]{mcode}  % To load matlab code
\lstset{deletestring=[b]{"}}

\usepackage{fancyhdr} % Custom headers and footers
\pagestyle{fancyplain} % Makes all pages in the document conform to the custom headers and footers
\setlength\parindent{0pt}
\fancyhead[L]{SF3565, Project 4, December 2016}
\fancyhead[R]{H{\"a}ggmark, Karlsson} % Empty left footer
\fancyfoot[C]{Program construction in C++ for Scientific Computing} % Empty center footer
\fancyfoot[R]{\thepage} % Page numbering for right footer

\title{Program construction in C++ for Scientific Computing \\ Teacher: Michael Hanke}

\author{Ilian H{\"a}ggmark \\ mail \href{mailto:ilianh@kth.se}{ilianh@kth.se}
  \and Andreas Karlsson \\ mail \href{mailto:andreas.a.karlsson@ki.se}{andreas.a.karlsson@ki.se} }
\date{\normalsize\today} % Today's date or a custom date

\begin{document}

\maketitle % Print the title

\section*{Project 4}
\subsection*{Task 1 - Redesign Domain class}

The Domain class is given a few new short methods to enable access to variables from the outside. It is moreover split up in a source file and header file. Inlining is also added explicitly. The array sides which contains the boundaries for the grid is rewritten so it uses smart pointers (\texttt{shared\_ptr}).

\subsection*{Task 2 - GFkt class}

The \texttt{GFkt} class holds two important objects. A \texttt{Domain} object that describes the grid and a \texttt{matrix} object that contain function values of a function \texttt{fp} on the grid. Apart from the basic structure with constructor/copy-constructor etc., the \texttt{GFkt} class contains methods to perform basic (point-wise) arithmetic operation on the matrix containing the function values. Finally the class have methods for calculating approximate discrete partial derivatives. \\

The first partial derivatives in $x$ and $y$ are calculated as.

$$ \frac{\partial}{\partial x} u(x_i,y_j)  = \frac{u(x_{i+1},y_j)-u(x_{i-1},y_j)}{x_{i+1} - x_{i-1}} \qquad  \frac{\partial}{\partial y} u(x_i,y_j)  = \frac{u(x_{i},y_{j+1})-u(x_{i},y_{j-1})}{y_{j+1} - y_{j-1}}$$

These are central derivatives that yield good accuracy, but they cannot be applied to all borders (first and last column for $x$ and firsts and last row for $y$). We therefore need one-sided derivatives to estimate the border derivatives.

$$ \frac{\partial}{\partial x} u(x_i,y_j)  = \frac{u(x_{i+1},y_j)-u(x_{i},y_j)}{x_{i+1} - x_{i}} \qquad  \frac{\partial}{\partial y} u(x_i,y_j)  = \frac{u(x_{i},y_{j+1})-u(x_{i},y_{j})}{y_{j+1} - y_{j}}$$

This would be for the first column and first row respectively. The one-sided derivatives have lower accuracy compared to the central derivatives. We therefore derive two a simple expression for a one-sides approximate derivative on a non-equidistant grid with Taylor expansion.


$$u(x+h_1) = u(x) + u'(x)\cdot h_1 + \frac{1}{2}u''(x)\cdot h_1^2 + \mathcal{O}(h^3)$$
$$u(x+h_2) = u(x) + u'(x)\cdot h_2 + \frac{1}{2}u''(x)\cdot h_2^2 + \mathcal{O}(h^3)$$

Multiply first equation with $h_2^2$ and the second with $-h_1^2$. Then sum is them:

$$ h_2^2 u(x+h_1)  - h_1^2 u(x+h_2)  = u(x) (h_2^2 - h_1^2 ) + u'(x) (h_1 h_2^2 - h_2 h_ 1^2) +  \mathcal{O}(h^3) $$

We neglect the higher order terms and solve for $u'(x)$

$$  u'(x) = \frac{- u(x) (h_2^2 - h_1^2 ) + h_2^2 u(x+h_1)  - h_1^2 u(x+h_2) }{h_1 h_2^2 - h_2 h_ 1^2}  $$

This 3-stencil approximation is marginally more complicated but gives much high accuracy. \\


For the Laplacian, $\Delta$, we can use a similar approach. The Laplacian of a function $u(x,y) $ is simply $\frac{\partial^2}{\partial x^2} u(x,y) + \frac{\partial^2}{\partial y^2} u(x,y) $. The two second order derivatives can be calculated separately and then added together. Each second order derivative is calculated with a central difference,

$$ \frac{\partial^2}{\partial x^2} u(x_i,y_j)  = \frac{u_x(x_{i+1},y_j)-u_x(x_{i-1},y_j)}{x_{i+1} - x_{i-1}} \qquad  \frac{\partial^2}{\partial y^2} u(x_i,y_j)  = \frac{u_y(x_{i},y_{j+1})-u_y(x_{i},y_{j-1})}{y_{j+1} - y_{j-1}}$$

where the only difference from the first order derivatives is that here the first order derivative is the input instead of the function. Also here we get the problem with border values and one-sided derivatives derived with Taylor expansion are used again. This time however we use a 4-stencil to achieve enough accuracy.

$$u(x+h_1) = u(x) + u'(x)\cdot h_1 + \frac{1}{2}u''(x)\cdot h_1^2 + \frac{1}{6}u'''(x)\cdot h_1^3 + \mathcal{O}(h^4)$$
$$u(x+h_2) = u(x) + u'(x)\cdot h_2 + \frac{1}{2}u''(x)\cdot h_2^2 + \frac{1}{6}u'''(x)\cdot h_2^3 + \mathcal{O}(h^4)$$
$$u(x+h_3) = u(x) + u'(x)\cdot h_3 + \frac{1}{2}u''(x)\cdot h_3^2 + \frac{1}{6}u'''(x)\cdot h_3^3 + \mathcal{O}(h^4)$$

We eliminate the first order and third order derivatives, neglect the high order terms and solve for the second order derivative. After some simple, but somewhat lengthy algebraic manipulations we get

\begin{align*}
u''(x) &= 2\cdot [ -u(x)((h_3^3 - h_1^3)(h_2h_3^3 - h_3h_2^3)-(h_3^3-h_2^3)(h_1h_3^3-h_3h_1^3))  \\
& + (h_2h_3^3 - h_3h_2^3)(h_3^3u(x+h_1)-h_1^3u(x+h_3))  - (h_1h_3^3 - h_3h_1^3 )(h_3^3 u(x+h_2) -h_2^3 u(x+h_3)) ]\\
&  /  [(h_1^2 h_3^3 - h_3^2 h_1^3 )(h_2 h_3^3 - h_3 h_2^3 )-( h_2^2 h_3^3 - h_3^2 h_2^3 )(h_1 h_3^3 - h_3 h_1^3) ]
\end{align*}

This expression gives a reasonable accuracy (depends on application of course).





%The discrete differential operator $\frac{\partial}{\partial x}$  can be approximated as simply $u(x+1,y)-u(x-1,y)$ and analogously for $y$. The second order %derivative can be approximated with $u(x+1,y)-2 \cdot u(x,y) + u(x-1,y)$ for the $x$-direction. The Laplacian can be expressed as

%$$
% \begin{bmatrix}
%  0 & 1 &  0 \\
%  1 & -4 &  1 \\
%  0 & 1 &  0
% \end{bmatrix}
% $$
%
% The code below shows these approximate derivatives in \textsc{Matlab}:
%
%\begin{lstlisting}
%% Approximate partial derivatives
%
%x = -30:0.1:30;
%y = -5:0.1:5;
%
%ux = sin((x/10).^2).*cos(x/10); % x-comp of u(x,y)
%[X,Y] = meshgrid(ux,y);
%u = X+Y;
%
%Dx = zeros(size(u));    % first partial derivative in x
%Dy = zeros(size(u));    % first partial derivative in y
%L = zeros(size(u));     % laplacian
%
%for i = 2:(length(x)-1)
%    for j = 2:(length(y)-1)
%        Dx(j,i) = u(j,i+1)-u(j,i-1);
%        Dy(j,i) = u(j+1,i)-u(j-1,i);
%        L(j,i) = -4*u(j,i)+u(j-1,i)+u(j+1,i)+u(j,i-1)+u(j,i+1);
%    end
%end
%\end{lstlisting}

\subsection*{Task 3 - Discrete differential operators}

The function $u(x,y)$ and its derivatives are shown below

\begin{align*}
u(x,y) &= \sin\left(\left(\frac{x}{10}\right) ^2\right ) \cos(x/10)+y \\
\frac{\partial}{\partial x} u(x,y) &= \frac{2}{100}x\cos\left(\left(\frac{x}{10}\right) ^2\right )\cos\left(\frac{x}{10}\right )- \frac{1}{10}\sin\left(\left(\frac{x}{10}\right) ^2\right )\sin\left (\frac{x}{10}\right )\\
\frac{\partial}{\partial y} u(x,y) &= \frac{\partial}{\partial y}y = 1   \\
\frac{\partial^2}{\partial x^2} u(x,y) &= \frac{2}{100}\cos\left(\left(\frac{x}{10}\right) ^2\right )\cos\left(\frac{x}{10}\right ) - \left ( \frac{2}{100}x\right )^2 \sin \left(\left(\frac{x}{10}\right) ^2\right )\cos\left(\frac{x}{10}\right ) \\
 &- \frac{1}{250}x\cos\left(\left(\frac{x}{10}\right) ^2\right )\sin\left(\frac{x}{10}\right ) - \frac{1}{100}\sin\left(\left(\frac{x}{10}\right) ^2\right )\cos\left (\frac{x}{10}\right )\\
\frac{\partial^2}{\partial y^2} u(x,y) &= 0 \\
\Delta u(x,y) &= \frac{\partial^2}{\partial x^2} u(x,y) + \frac{\partial^2}{\partial y^2} u(x,y) =  \frac{\partial^2}{\partial x^2} u(x,y)
\end{align*}


\newpage

The code below plots the algebraic expressions in \textsc{Matlab}. Figure \ref{fig:algebraic} shows the output.

\begin{lstlisting}
% Algebraic expressions of first and second partial derivatives in x

x = linspace(-10,5,50);		     % length set to 50 to agree with data from GFkt

figID = figure(101);                              % open figure

ux = sin((x./10).^2).*cos(x./10);                 % x-component of u(x,y)

dxu = 2/100.*x.*cos((x./10).^2).*cos(x./10) ...   % first partial derivative
      - 1/10*sin((x./10).^2).*sin(x./10);

dxxu = 2/100*cos((x./10).^2).*cos(x./10) ...      % second partial derivative
       - (2/100.*x).^2.*sin((x./10).^2).*cos(x./10) ...
       - 1/250*cos((x./10).^2).*x.*sin(x./10) ...
       - 1/100*sin((x./10).^2).*cos(x./10);

subplot(1,3,1); plot(x,ux);
subplot(1,3,2); plot(x,dxu);                      % plot result in subplot
subplot(1,3,3); plot(x,dxxu);

\end{lstlisting}


\begin{figure}[H]
  \centering
  \includegraphics[width=0.8\textwidth]{algebraicDeriv.pdf}
  \caption{\small Plot of algebraic expressions in \textsc{Matlab}: (a) $x$-component of $u(x,y)$. (b) First partial derivative of $u(x,y)$ in $x$. (c) Second partial derivative of $u(x,y)$ in $x$.\label{fig:algebraic}}
\end{figure}

With the approximate derivatives implemented in task 2 we can now plot the algebraic derivatives alongside the C++ approximations (Fig \ref{fig:1derivX} and \ref{fig:2derivX}).
% The error is shown in blue and defined as.
%$$\textrm{relative error} = 2\cdot \frac{u_\textrm{algebr} - u_\textrm{approx}}{u_\textrm{algebr} + u_\textrm{approx}}$$
We note with no surprise that the approximate derivatives have a good agreement with the algebraic derivatives in all places except the borders. The agreement is so good that the green line showing the approximation is hard to discern from the red line. We therefore increase the thickness of the green line and add a blue error line with a separate axis. As stated above accuracy is not an absolute in real applications, but must be set in relation to something else. We can however say that current accuracy should be good enough for a first rough estimate when solving a PDE.


\begin{figure}[H]
  \centering
  \includegraphics[width=0.8\textwidth]{comparison-x.pdf}
  \caption{\small Algebraic first order x-derivative (red), approximation (green), and error (blue).\label{fig:1derivX}}
\end{figure}



\begin{figure}[H]
  \centering
  \includegraphics[width=0.8\textwidth]{comparison-xx.pdf}
  \caption{\small Algebraic second order x-derivative (red), approximation of Laplacian (green), and error (blue). The Laplacian can be compared to the second order derivative in $x$ since the $y$ derivative is several orders of magnitude smaller (algebraically zero). \label{fig:2derivX}}
\end{figure}


\end{document}
